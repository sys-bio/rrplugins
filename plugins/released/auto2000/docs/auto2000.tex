\chapter*{Bifurcation using AUTO2000 and the Auto2000 Tellurium Plugin}
\setcounter{chapter}{1}
\emph{Continue with some bifurcation}
\section{Introduction}
The AUTO2000 plugin serves as a front-end for the AUTO2000 library, which is a library for continuation and bifurcation problems in ordinary differential equations 
\footnote{AUTO2000 by 
Eusebius J. Doedel , Randy C. Paffenroth, Alan R. Champneys, Thomas F. Fairgrieve, Yuri A. Kuznetsov, Bart E. Oldeman, Bj\"orn Sandstede and Xianjun Wang.
See http://www.dam.brown.edu/people/sandsted/publications/auto2000.pdf.
}.

Current limitations: Multiple continuation parameters are not supported, i.e. only one parameter can be selected for any continuation problem.

Available properties in the auto2000 plugin are documented in the next section.

\begin{landscape}
\section{Plugin Properties}
The AUTO library has numerous properties that have been wrapped and made available to a plugin client. Each property is listed below with its data type, default value and 
a short description. For the exact usage and a more in detail description please consult the main AUTO2000 manual.

%\begin{table}[ht]
\centering % used for centering table
\begin{longtable}{p{4cm} l p{3cm}  p{10cm}} % centered columns 

Property Name & Data Type & Default Value  & Description \\ [0.5ex] % inserts table 
%heading
\hline % inserts single horizontal line
SBML                                &   string              & N/A     &   SBML document as a string. Model to be used by AUTO \\
TempFolder                          &   string              & "."     &   Folder used by auto and the plugin for saving temporary files \\
KeepTempFiles                      &   bool                & false    &   Boolean indicating if temporary files should be deleted after an AUTO session or not \\
ScanDirection                      &   string              & "Positive" &   Parameter instructing AUTO how to sweep its principal continuation parameter \\
PrincipalCont\-inuationParameter   &   string              & N/A      &   The principal continuation parameter (PCP) is the first parameter that AUTO will sweep. Currently only one parameter is \
                                                                            supported, which by default is the PCP \\
BifurcationPoints                  &   vector$<$int$>$     & N/A      &    This integer vector holds the exact point number (in the sequence of all output data) for an AUTO solution point. It can be used together with the labels in the bifurcationlabels property to assist in 
                                                                           plotting a bifurcation diagram \\
BifurcationLabels                  &   stringList          & N/A      &    The string list holds the AUTO designated solution type label for a solution point, as found in the bifurcationpoints property. Consult the AUTO documentation for
                                                                           possible label types and their meaning \\
BifurcationData                    &   telluriumData       & N/A      &    The  BifurcationData property holds the bifurcation diagram after a session. The first column contains the values of the selected parameter, and successive columns are selected species \\


\\[2pt]                                                               
\multicolumn{4}{p{19cm}}{The following properties are used internally by the auto library. Depending on the problem at hand, they may need to be tweaked. } \\[12pt]
\hline %inserts single line                                           
\\[2pt]                                                               
fort2                              &   string             & N/A      &    Property containing the content of the AUTO temporary file, fort.2. Fort.2 is the input file for AUTO and created by the plugin \\
fort3                              &   string             & N/A      &    Property containing the content of the AUTO temporary file, fort.3. The content of fort.3 file is undocumented in AUTO's documentation \\
fort6                              &   string             & N/A      &    Property containing the content of the AUTO temporary file, fort.6. The content of fort.6 file is a bifurcation session summary \\
fort7                              &   string             & N/A      &    Property containing the content of the AUTO temporary file, fort.7. The content of fort.7 file is a bifurcation diagram on success \\
fort8                              &   string             & N/A      &    Property containing the content of the AUTO temporary file, fort.8. The content of fort.8 file contain various statistics from the session \\
fort9                              &   string             & N/A      &    Property containing the content of the AUTO temporary file, fort.8. Diagnostic messages, convergence history, eigenvalues, and Floquet multipliers 
                                                                            are written in fort.9 \\
NDIM                               &   int                 & 1        &    The NDIM property correspond to the dimension of the system of equations   \\
IPS                                &   int                 & 1        &    Constant defining the problem type \\
IRS                                &   int                 & 1        &    This constant sets the label of the solution where the computation is to be restarted. \\
ILP                                &   int                 & 1        &    Fold detection; 1=ON, 0=OFF  \\
NICP                               &   vector$<$int$>$     & N/A      &    Property denoting the number of free parameters  \\
ICP                                &   int                 & N/A      &    Free parameters  \\
NTST                               &   int                 & 15       &    The number of mesh intervals  \\
NCOL                               &   int                 & 3        &    The number of collocation points per mesh interval  \\
IAD                                &   int                 & 3        &    Mesh adaption every IAD steps; 0=OFF  \\
ISP                                &   int                 & 1        &    Bifurcation detection; 0=OFF, 1=BP(FP), 3=BP(PO,BVP), 2=all  \\
ISW                                &   int                 & 1        &    Branch switching: 1=normal, -1=switch branch (BP, HB, PD), 
                                                                           2=switch to two-parameter continuation (LP, BP, HB, TR) 3=switch to three-parameter continuation (BP)  \\
IPLT                               &   int                 & 0        &    This constant allows redefinition of the principal solution measure, which is printed as the second 
                                                                           (real) column in the fort.7 output-file. See AUTO manual for possible settings  \\
NBC                                &   int                 & 0        &    Number of boundary conditions  \\
NINT                               &   int                 & 0        &    Number of integral conditions  \\
NMX                                &   double              & 1000     &    Maximum number of steps  \\
RL0                                &   double              & 0.01     &    The lower bound on the principal continuation parameter  \\
RL1                                &   double              & 30       &    The upper bound on the principal continuation parameter  \\
A0                                 &   double              & 0        &    The lower bound on the principal solution measure  \\
A1                                 &   int                 & 10000    &    The upper bound on the principal solution measure  \\
NPR                                &   int                 & 50       &    Save the solution in the solution file every NPR continuation steps   \\
MXBF                               &   int                 & -1       &    Automatic branch switching for the first MXBF bifurcation \
                                                                           points if IPS=0, 1   \\
IID                                &   int                 & 0        &    Control diagnostic output; 0=none, 1=little, 2=normal, 4=extensive   \\
ITMX                               &   int                 & 8,       &    Maximum number of iterations for locating special solutions/points   \\
ITNW                               &   int                 & 5,       &    Maximum number of correction steps   \\
NWTN                               &   int                 & 3,       &    Corrector uses full newton for NWTN steps   \\
JAC                                &   double              & 0,       &    User defines derivatives; 0=no, 1=yes   \\
EPSL                               &   double              & 1e-8     &    Property setting the convergence criterion for parameters   \\
EPSU                               &   double              & 1e-8     &    Property setting the convergence criterion for solution components   \\
EPSS                               &   double              & 1e-6     &    Property setting the convergence criterion for special points   \\
DS                                 &   double              & 0.001    &    Session start step size   \\
DSMIN                              &   double              & 1e-5     &    Minimum continuation step size   \\
DSMAX                              &   double              & 0.1      &    Maximum continuation step size   \\
IADS                               &   int                 & 1        &    Step size adaption every IADS steps; 0=OFF   \\
NTHL                               &   int                 & 0        &    The number of modified parameter �weights� (for BVP)   \\
THL                                &   vector$<$int$>$     & N/A      &    List of parameter weights   \\
NTHU                               &   int                 & 0        &    The number of modified solution component �weights� (for BVP)   \\
THU                                &   vector$<$int$>$     & N/A      &    List of solution weights  \\
NUZR                               &   int                 & 0        &    The number of �user output points� specified   \\
UZR                                &   vector$<$int$>$     & N/A      &    List of values for user defined output \\

                                                        
\hline %inserts single line                             
\caption{Plugin Properties} 
\label{table:autoPluginProperties} 
\end{longtable}
%\end{table}

\end{landscape}

\section{The \texttt{execute(bool inThread)} function}
The \verb|execute()| function will start a bifurcation session. Depending on the problem at hand, the algorithm may run for a long time. 

The \verb|execute(bool inThread)| method supports a boolean argument indicating if the execution of the plugin work will be done in a thread, or not.
If set to false, i.e. executing \verb|execute(false)|, the function will be a blocking function and will not return until the plugin work is done. If it is set
to true, the \verb|execute(true)| will return immediately and the plugin work will be executed in 
a thread. The user can use the \verb|isPluginDone(plugin)| to query the status of the plugin progression.

The inThread argument defaults to \textbf{false}. 


\section{Plugin Events}
The auto2000 plugin uses all of the available plugin events, i.e. the \emph{PluginStarted}, \emph{PluginProgress} and the \emph{PluginFinished} events.

The available data variables for each event are internally treated as \emph{pass through} variables, so any data, for any of the events, assigned prior to 
the plugins execute function (in the assignOn() family of functions), can be retrieved \emph{unmodified} in the corresponding event function.

\begin{table}[ht]
\centering % used for centering table
\begin{tabular}{l l p{9cm}} 

Event & Arguments & Purpose and argument types \\ [0.5ex] % inserts table 
%heading
\hline % inserts single horizontal line
PluginStarted  	& 	void*, void*  & Signal to application that the plugin has started. Both parameters are \emph{pass through} parameters and are unused internally by the plugin.\\[0.5ex]
PluginProgress	& 	void*, void*  & Communicating progress of fitting. Both parameters are \emph{pass through} parameters and are unused internally by the plugin. \\[0.5ex]
PluginFinished	& 	void*, void*  & Signals to application that execution of the plugin has finished. Both parameters are \emph{pass through} parameters and are unused internally by the plugin.\\

\hline %inserts single line
\end{tabular}
\caption{Plugin Events} 
\label{table:autoPluginEvents} 
\end{table}

\section{Python example}
The following Python script illustrates how the auto plugin can be invoked, how to set its properties and finally how to plot a bifurcation diagram. 

\begin{singlespace}
\lstinputlisting[label=AutoPluginExample,caption={Bifurcation example using a complex model.},language=Python]{Examples/telAUTO2000.py}
\end{singlespace}

\newpage
\begin{figure}
\centering
\includegraphics[width=160mm]{AUTOOutput.png}
\caption{Output for the example script above.}
\label{fig:nmFig}
\end{figure}






